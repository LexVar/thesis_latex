% #############################################################################
% This is Chapter 5
% !TEX root = ../main.tex
% #############################################################################
% Change the Name of the Chapter i the following line
\fancychapter{Implementation}
\cleardoublepage
% The following line allows to ref this chapter
\label{chap:implementation}

This chapter presents the details of the developed solution on the Hardware Security Module, the user software and the developed API, to access the board's services.
It covers the implementation details, namely the standards and libraries, the developed services, their logic and how they leverage the board's existing accelerators. The principal components detailed are the communications, secure data exchange service, digital signatures, key management solution and tamper detection features.

% -----------------------------------------------------
\section{Libraries and Tools}\label{chap:implementation:tools}

The device's software is running on the \ac{SoC} of a SmartFusion2 board, version M2S090TS from Microsemi, with the features described in section~\ref{chap:background:computing:smartfusion}.
The application was implemented using the C programming language, with the SoftConsole v3.4 \ac{IDE} and drivers to access the board's functionalities, provided by Microsemi. The board's features were configured using Libero v12.2.
The device functions as a \ac{HSM}, connected through a \ac{USB} connection to a computer. It was programmed using the external FlashPro4 programmer required to develop and debug embedded applications with SoftConsole \cite{smartfusionSecurityPractices}.
An API was implemented using the \ac{PKCS}\#11 standard for a higher device interoperability of cryptography systems. A user application, with a simple CLI interface, which invokes the PKCS functions was implemented. Both this software and the API were implemented in C++, for the open source MinGW compiler for Windows 10.
The computer running the user software, and the smartfusion board running the HSM code, are connected through a \ac{USB} connection.
The device implementation uses the available \ac{UART} communication controller and corresponding drivers for communications, while the user application uses the windows drivers to communicate with the corresponding serial port.
This setup allows the board to send and receive messages of 16 byte blocks at a time.
% Mbedtls consideration
% It is an adequate device due to several components such as the required cryptographic functions in its system controller, a \ac{TRNG} essential for cryptography, secure storage for keys and anti-tampering protections.
% The inclusion of a cryptography library, such as Mbedtls, designed for embedded systems was considered. The library provides algorithm implementations: ECDH, ECDSA, AES encryption and HMAC. The library can be integrated with the boards functionalities and have unnecessary features disabled to lower the used memory.
% Ultimately, the security services offered by the device provide the necessary functionality to fulfill the requirements defined previously, with the exception of generating the device's certificate, but this was done by exporting the public key, generating the certificate and loading the certificate on the device.
% Beyond increasing the memory footprint, the library does not offer significant performance benefits, or security measures against side-channel attacks similar to the one's offered by the device's services.

 % \cite{parrinha2017flexible}
% The paper compares the performance of operations on the smartfusion2 with the mbedtls library and the board's security cores. It concludes AES encryption has a higher throughput of 1.8 Mbit/s on the mbedtls implementation compared to the 730 Kbit/s on the embedded core. For the elliptic curve scalar multiplication, used for ECDH, the software implementation is significantly slower, from 28.4s per operation compared to 0.57s with the embedded core.

% -----------------------------------------------------
\section{HSM Services}\label{chap:implementation:services}

The services were implemented using the SmartFusion2 security cores: SRAM-PUF, AES, HMAC, SHA, ECC, TRNG and tamper detection.
All these services are accessed through the developed PKCS\#11 API. Its communication protocol is detailed in section \ref{chap:arch:services}.
% RAM
The eSRAM 64 KB is reserved for code. For each byte of RAM, there are 2 bits for error detection and correction, a total of 16 KB. This feature mitigates \ac{SEU}. It corrects 1 bit errors and detects up to 2 bit errors. If memory is essential for more application code or buffers, the board can be configured with this setting disabled in Libero. This frees the additional 16 KB of memory for RAM, totaling 80 KB.

% -----------------------------------------------------
\subsection{Login}\label{chap:implementation:services:authentication}

The login operation can always be be accessed by the user, and authenticates the user using a 8 digit PIN. All other implemented services can only be accessed after the user is logged in.
The authentication PIN is saved in the SRAM-PUF service in a predefined and static slot, or using the developed key management solution detailed in section \ref{chap:implementation:services:key-import}. When authentication is performed, the PIN is generated and read, compared with the client's supplied PIN and erased from memory.
If the user is authenticated, the PIN can be changed, by overwriting the existing value.

\subsection{Secure Communications}\label{chap:implementation:services:secure}

The goal of this service, is to provide authentication and confidentiality to communications.
As previously mentioned, the security of \ac{AES}-\ac{GCM} in hardware is considered to be unsurpassed by any other \ac{AES} authenticated-encryption scheme~\cite{aesmodes}.
Unfortunately, the SmartFusion2 board, does not support this encryption algorithm, or any authenticated-encryption scheme.
Therefore the devised solution combines an \ac{AES} encryption mode with the authentication algorithm \ac{HMAC}. Among the supported \ac{AES} modes, as concluded in chapter~\ref{chap:background}, CTR mode is the most favourable option because of its efficiency and security, without the added complexity of implementing a padding solution to the secure block mode \ac{CBC}.
For authentication, the board supports \ac{HMAC} with the \ac{SHA}-256 hash function, which uses a 256 bit symmetric key and generates a 256 bit code. As concluded in section \ref{chap:background:crypto:mac}, it is a well-designed construction, even though it is not the most efficient.
For combining these algorithms, studies recommend combining a secure encryption and secure \ac{MAC} with the encrypt-then-MAC method~\cite{encryptmacorder}.
The key used in HMAC should be different from the one used in encryption, to ensure the best security practices, by not reusing the same key in different algorithms. So in practice, a key used for securing communications is split in two keys, one for encryption and one for authentication.
When choosing key sizes, taking into account the limited storage capacity of the board, a smaller, but still secure, key size is preferred. The \ac{AES} 128 bit and 256 bit services guarantee 128 and 256 bit security, respectively. HMAC with SHA-256 provides 256 bit security, with 256 bit keys. According to the \ac{NIST} recommendations~\cite{nistRecommendations}, algorithms which guarantee both 128 and 256 bit security, are expected to be secure from 2031 and beyond. If storage is extremely limited, AES with 128 bit keys is a secure and adequate option. However, with 256 bit security, the system will offer a longer life expectancy, which can be convenient for not easily updatable systems.
With AES 128 bit, the secure data service provides 128 bit security, since it has the lowest security level compared to HMAC which provides 256 bit. Choosing AES with 256 bit keys, provides 256 bit security to secure communications.
In the developed implementation, for every communication channel, two 256 bit keys are used.

For every new encryption operation, a unique 16 byte \ac{IV} is generated, with the \ac{NRBG} service.
Then the plaintext, sent by the user software, is encrypted, and the MAC is generated from the ciphertext and IV. The final output of IV, ciphertext and MAC is returned to the user.
The board was configured with its maximum of RAM, 80 KB. With all the code and drivers needed for this implementation, only around 36 KB of space for data buffers is left. Initially, two buffers of 18 KB each, were used, one for the input data, another for the result. With some improvements and tests, this was improved to a single 36 KB buffer, where the initial input was stored, and the result as well, while performing the cryptographic operations. With this implementation, the system is limited to securing 36 KB of data. The next logical upgrade, was to implement continuous cipher and authentication, while chunks of the total data were received from the user, one after the other. However the board is not capable of continuous operations, using its security cores.
\begin{figure}[h!]
	\centering
	\includegraphics[width=0.75\textwidth]{./Images/data-exchange-chunks.png}
	\caption{Communication protocol for continuous data ciphering and authentication in the HSM with internal keys}
	\label{fig:protocol:data-exchange-chunks}
\end{figure}
A solution was devised by taking advantage of the AES \ac{CBC} mode's characteristics and a lightweight implementation of continuous HMAC. Its protocol is depicted in figure \ref{fig:protocol:data-exchange-chunks}. The mode starts encryption, with a random IV for the first block. The IV for all the subsequent blocks, is the generated ciphertext of the previous block. Therefore, chunks can be encrypted with this mode, using the last ciphertext block of the last chunk, as the IV for the current chunk. The user software sends the chunks one at a time, the HMAC calculation is updated for each block, and the result ciphertext is returned each time. When there is no more plaintext to receive, the board returns the remaining ciphertext and final MAC value. The total output is composed of the initial IV, ciphertext and MAC.
This approach is more flexible, compared to knowing the total data size, before receiving the chunks. It does not limit the system, if used for example in a network setting. Data can be sent to the device, as it is received, and if the transmission is congested or broken, the system is not stuck waiting for the rest. The downside of this approach, is the internal buffer on the HSM must be managed, which adds implementation complexity.
As concluded previously, CBC is a secure mode, assuming a secure padding scheme is used to prevent padding oracle attacks. To avoid padding, and guarantee the mode's security, ciphertext stealing was implemented.

% Thus \ac{AES} with 128 bit keys is preferred, since it guarantees adequate security for the foreseeable future, and is the more conservative option.
% Even though the HMAC service needs a 256 bit key, a 128 bit key can be used with the upper 128 bits padded with zeros \cite{smartfusionSecurityPractices}.
% Therefore for every communication a 256 bit key is used, the lower 128 bits for encryption and upper 128 bits for authentication.

% The encryption and authentication keys, both 128 bit, are fetched from the SRAM-PUF service, using the slot (key ID) requested by the client application.
% The returned string is either the original plaintext, or the \ac{MAC}, \ac{IV} and ciphertext, concatenated in that order.

% --------

% The system provides a single hash algorithm, \ac{SHA}-256, which has a security strength of 128 bits.
% The \ac{HMAC}-\ac{SHA}-256 algorithm needs a 256 bit key, different from the one used with encryption, to ensure the best security practices. A 128 bit key can be used, if padded with zeros. Thus, for every communication, a 128 bit key for encryption and a 128 bit key for authentication is used. In total, a 256 bit shared secret.

% The qualified digital signatures operation combines \ac{ECC} and the \ac{SHA}-256 hash algorithm, and provides 128 bit security, calculated by taking the smallest security strength value of both algorithms. For the encryption and authentication operation combining \ac{AES}-128 and \ac{HMAC}-\ac{SHA}-256, 128 bit security is provided.

% -----------------
\subsection{Digital Signatures}\label{chap:implementation:services:signatures}

The board supports asymmetric keys, on the \ac{ECC} P-384 curve. The board has three \ac{ECC} core functions, multiplication of a point with a scalar value, addition of two points on the curve and one for obtaining the curve's base point. The scalar multiplication, allows the generation a public key, which is a point of 96 bytes, from the multiplication of the corresponding private key, a 48 byte scalar value, and the curve's base point. Scalar multiplication can also be used to multiply another entity's public key, with the personal private key, to generate a shared secret. This is the logic behind the \ac{ECDH} algorithm.
The board is not equipped with a \ac{ECDSA} API. The algorithm can be implemented, by combining the provided ECC security cores, the SHA core, and a big numbers library, which allows operations between numbers of at least 48 bytes.
A big numbers library was included in the implementation for this purpose. The implementation was successfully tested, with a notable drawback.
The library takes up 54 KB of RAM space, striped of every unnecessary function and code. Thus, this functionality only works if all other implemented features are disabled, with a limited buffer for data. A continuous SHA implementation could be used, to overcome this limitation.
This is an extremely limited use case, and less than ideal. Future work should revise this functionality, by evaluating existing lightweight libraries, which provide the necessary features, or even a custom implementation. The limitation could also be overcome by replacing the SHA core with a continuous SHA implementation. However, the RAM space for additional functionalities would still be limited.

% -----------------
\subsection{Key Management}\label{chap:implementation:services:key-import}

This section outlines the various key management options, their strengths, weaknesses and the implemented solution. 
As introduced in section \ref{chap:background:computing:smartfusion}, the SmartFusion2 includes a secure key storage service. The PUF has 56 available key slots, each with a maximum of 512 bytes. There is a lifespan problem regarding this service. The eNVM memory, used by the service, is limited to 1000 writes for a predicted lifespan of 20 years. Thus, there is a limit on the amount and frequency, of keys saved in the slots. The service should be used carefully, restraining how often it is written to.
\begin{figure}[h!]
	\centering
	\includegraphics[width=0.7\textwidth]{./Images/key-management.png}
	\caption{Key management and importation solution using PUF and encrypted keys in storage}
	\label{fig:implementation:key-management}
\end{figure}
An alternate solution, in which the keys are stored in a non volatile memory, was developed, and is pictured in figure \ref{fig:implementation:key-management}. The keys are stored encrypted and authenticated with a dedicated key, following the protocol in section \ref{chap:implementation:services:secure}.
The key which encrypts and authenticates all other keys, is an intrinsic key, randomly generated in the device, and stored in a dedicated PUF slot. The ciphertext and IV of the keys, is stored in memory, authenticated with HMAC. The output code is stored in a separate slot on the PUF.
A key can be accessed by generating the MAC of the stored data, comparing it to the one stored in the PUF slot. If authenticated, then the keys can be decrypted with the IV and dedicated PUF key.
While this solution still uses the PUF service, its usage is more measured. When using only the PUF core, for each key, one PUF slot needs to be rewritten. In the case of the implemented solution, for each change of the key set stored in memory, only the MAC PUF slot needs to be rewritten. If keys are added one at a time, both options are identical, in the amount of PUF slots written. However, if keys are updated on a regular time schedule, all can be updated at once, with each update only using one slot. Additionally, the amount of available key storage is not a problem, since the board allows for external storage devices to be connected, and keys can be securely stored there. Even if attackers get hold of the external storage, the encryption keys are protected in the SRAM-PUF.

To complement this solution, an import operation was implemented, which receives a set of encrypted keys, decrypts them with the secure data exchange service, and overwrites the existing keys. This can be used, in conjunction with a control station, which distributes the keys to users with these devices. The user's can import the key set, which overwrites the existing keys.
Each key is stored in memory accompanied by its identifier and size.

As introduced in section \ref{chap:background:computing:smartfusion}, the AES, HMAC and SHA cores only have light measures against \ac{DPA}. Thus, keys should be regularly replaced, to avoid attackers getting hold of enough data to break encryption keys. On the other hand, this is only possible if the attacker has physical access to the device. 
This type of threats can be mitigated using the import keys service, to regularly replace encryption keys.

% -----------------
\subsection{Key Generation}\label{chap:implementation:services:key-generation}

The \ac{ECC} P-384 curve guarantees 192 bit security.
The key generation, follows the protocol and guidelines defined in the previous chapter \ref{chap:arch:services:new-comms:ecdh}.
The peer's public key is supplied by the user application, and the device's private key is fetched from a predetermined PUF slot, or from memory.
For the ECDH algorithm, the ECC scalar multiplication board accelerator is used. It multiplies a scalar value of 48 bytes, the private key, with a point of 96 bytes, the peer's public key. The result is a 96 byte point, which is effectively the shared secret.The same shared secret can be computed by the peer in their device, using their private key, and this user's public key.
The shared secret can be used directly as the new key, or can be inputted in a key derivation function.
The implementation derives a 256 bit key, with the KeyTree service. It is used as a key derivation function on the shared secret, combined with a salt, which can be public.
By applying a key derivation function with a public salt, multiple new keys can be generated from the same public key pairs, by supplying different salts. 
The key can be stored in the PUF or encrypted in memory. When stored, its ID is returned to the user.

% -----------------
\subsection{Tamper Detection}\label{chap:implementation:services:tamper-detection}

Tamper attacks and possible attempts can be detected by the board. When these events occur, flags are asynchronously set, which warn the user from performing operations, to protect from possible attacks.
On tamper attempt events, the board's leds are turned on to warn the user, and the device software is blocked from accepting any more commands.
When the tamper attacks are flagged, zeroization is performed on the device, erasing all keys, memory and data from the device.
The process can place the device in three possible states. It can be permanently unusable, reset to its initial state from fabric or recoverable only with a new key file supplied by microsemi. Zeroization was tested in the mode which resets the board to its fabric state.
These measures can have a denial-of-service effect on the system, but is a trade-off deemed necessary, in order to avoid successful attacks and potential leaks of sensitive information.

% -----------------------------------------------------
\section{PKCS\#11 API}\label{chap:implementation:app:pkcs}

An PKCS\#11 API was developed to access the HSM services. It is invoked by the user software.
The implementation only supports one token and one session for the token.
PKCS\#11 defines two types of users, the regular user and the security officer. The implemented application does not make that distinction. Both types of user can login, and gain access to all other services.
The full list of the implemented PKCS\#11 functions, is listed in table \ref{tab:pkcs11-api}.

% -----------------------------------------------------
% -----------------------------------------------------
\section*{Summary}\label{chap:implementation:summary}

This chapter presented the implementation details, of the prototype, developed on the SmartFusion2 board. It specified the libraries and tools used, all implemented services details and rationale, as well as the developed PKCS\#11 API to access these services. It focused on secure communications with authentication and confidentiality. It provided improvements on the baseline set in chapter 4, trade offs and benefits of including some software implementations. It also provided flexibility with a key management solution, using the board's services efficiently, new key generation and digital signatures using the board's public-key cryptography features. Lastly, it presented the available tamper detection options and zeroization features, providing an implementation example for both.

% The user software initializes, the cryptoki application and opens the session with the device through the \texttt{C\_Initialize} and \texttt{C\_OpenSession} functions.
% The opposite \texttt{C\_CloseSession} and \texttt{C\_Finalize} functions close the session, and terminate the application.
% The user interface handles the input from the user, and uses the data to call the functions in the implemented PKCS\#11 API.
% The user inputs the operation code he wishes to perform, and the application handles the correct API calls to the PKCS\#11 interface.
% The user can authenticate itself by calling the \texttt{C\_Login} function. The authentication PIN can then be changed with the \texttt{C\_SetPIN} function.
% The \texttt{C\_Logout} function logs out the user.
% To encrypt and authenticate some data, a local object for the secret key is created with \texttt{C\_CreateObject}. It does not contain the key data, only the key ID, so it can be passed of to the PKCS\#11 API. Continuous encryption is performed through calls to \texttt{C\_EncryptInit} to initialize the operation, \texttt{C\_EncryptUpdate} for every chunk, and \texttt{C\_EncryptFinal} to end the operation.
% The same logic is applied to continuous decryption, with \texttt{C\_DecryptInit}, \texttt{C\_DecryptUpdate} and \texttt{C\_DecryptFinal}.
% For deleting a key on the device, first a secret key object is created with its ID, like before, and then a call to \texttt{C\_DestroyObject} erases the key from the device.
% A new key is generated with the ECDH mechanism and \ac{SHA}-256 key derivation function, with a call to \texttt{C\_DeriveKey}.
% Only one custom which is called directly is implemented. The \texttt{HSM\_C\_GetKeyList} returns the number of enrolled keys in the device.

% The solution should support a plethora of devices, as it will increase the adoptability of the solution among clients. This entails the use of a widely established protocol, which clearly defines a set of functions and standards the system should follow.
% The \ac{PKCS} \#11 standard will be utilized to fulfill this requirements. It allows operations to be standardized across different devices, increasing the range of supported devices. By implementing the system in accordance with these guidelines, it will have a higher device interoperability. Additionally, it allows the application to use, create and modify objects, without exposing them to its memory.
% % -----------------------------------------------------
% \subsection{Key Management Protocols}\label{chap:implementation:protocol:key}
%
% The protocols for the communication management operations are detailed next, namely for symmetric key generation. The protocol to generate a new symmetric key with another entity using asymmetric cryptography is detailed in figure~\ref{fig:protocol:ecdh}.
%
% \begin{figure}[h!]
%         \centering
%         \includegraphics[width=0.60\textwidth]{./Images/ecdh.png}
%         \caption{Communication protocol to generate symmetric keys, with the HSM internal private key, and stored them internally}
%         \label{fig:protocol:ecdh}
% \end{figure}
%
% % The device calls the ECDH algorithm with its private key and the received public key to generate a secret and run it through a key derivation function with the salt value to obtain the new symmetric key. The service's formula is thus:
% The user forwards the operation code, public key and salt value to the device.
% The device will then generate a symmetric key, using the process in equation \ref{eq:ecdh-kdf}.
%
% \begin{equation}
%         \label{eq:ecdh-kdf}
%         KDF\{ECDH_{K}\{K^{-1}\}, Salt\}
% \end{equation}
%
% A secret is generated from the \ac{ECDH} algorithm, with the device's private key, and another entity's public key. The same secret can be generated with the device's public key and the other entity's private key.
% This secret is run through a key derivation function in conjunction with a salt which can be public, to generate a symmetric key. If both entities generate the same secret and agree on the same salt value, the generated key is identical.
%
% The new key is then stored in non-volatile memory encrypted with the dedicated symmetric encryption key for storage, saved in its PUF slot.
% A new unique ID is generated for the key, which is stored alongside the key in memory. This ID is returned to the user along with the operation success status.
%
% % -----------------------------------------------------
% \subsection{Import Keys into HSM Protocol}\label{chap:implementation:protocol:import-keys}
%
% This section details the protocol to import a list of encrypted symmetric keys, into the device.
% The imported keys must be encrypted with a predefined symmetric key, stored in a dedicated PUF slot inside the device. This service is useful so a central management section with a similar device can distribute keys to the enrolled entities, on a regular schedule, such as monthly.
% The protocol for this service is pictured in figure \ref{fig:protocol:import-keys}.
%
% \begin{figure}[h!]
%         \centering
%         \includegraphics[width=0.60\textwidth]{./Images/import-keys.png}
%         \caption{Communication protocol to import encrypted symmetric keys, and store them encrypted in the HSM's memory}
%         \label{fig:protocol:import-keys}
% \end{figure}
%
% The only transmission consists in the client sending the operation code, keys length and key data.
% Then, the device decrypts and authenticates the keys using a predefined symmetric key stored in a secure PUF slot. The protocol used is the same for secure data exchange described in the previous section.
%
% Each key consists of 1 byte for its ID, 1 byte for its size and remaining bytes for the key data.
% The decrypted list of keys is encrypted again, using \ac{AES} \ac{CTR} mode with another predefined symmetric key for storage of data in non-volatile memory. The ciphertext, randomly generated IV and ciphertext length is stored in memory. To provide authentication and thus protection against tampering, a \ac{MAC} is generated from the three pieces of data and stored in a PUF slot.
%
% In order to fetch a stored symmetric key to use in one of the implemented services, the \ac{MAC} is generated and compared to the PUF slot. If identical, the key can be decrypted from memory using the same key used for its encryption and identified with its ID.

% All data is either stored in the secure eNVM portion, or in the non-volatile memory, encrypted with an internal \ac{KEK}.
