% #############################################################################
% This is Chapter 4
% !TEX root = ../main.tex
% #############################################################################
% Change the Name of the Chapter i the following line
\fancychapter{Architecture}
\cleardoublepage
% The following line allows to ref this chapter
\label{chap:arch}

The objective of the system was to develop a device in a box format to enable users to establish safe channels of communication. This is achieved with a safe and secure device which is personal to each individual. In order to secure the communications between users, the device saves the user's sensitive data, such as keys, and performs all security critical operations.
The system is designed so that each user has it's own physical box.

% -----------------------------------------------------
% -----------------------------------------------------
\section{Components}\label{chap:arch:components}

\begin{figure}[h]
    \centering
    \includegraphics[width=0.7\textwidth]{./Images/main-components.png}
    \caption{System Components}
    \label{fig:components}
\end{figure}

The system architecture, depicted in figure~\ref{fig:components}, is composed of two main components: the physical box which responsible for securing the data, and the client software on the user's computer, which communicates with the box.
The client software sends and receives data through the device's serial I/O port, which exposes an \ac{API} to access its operations.
With this connection the entity can signal the device to perform the desired operations, through the client software.
The device integrates a \ac{FPGA}, \ac{MSS}, secure eNVM storage and flash memory for configuration. The \ac{MSS} has an embedded ARM processor, connected to the system controller which provides several cryptographic services. The secure eNVM allows storage of keys, data and secure boot code.
The \ac{MSS} uses the keys stored in the eNVM, with the cryptographic algorithms in the system controller.

% -----------------------------------------------------
% -----------------------------------------------------
\section{Operations}\label{chap:arch:ops}

This section will define and describe the system architecture. It is structured starting with the authentication and then the system operations, divided by types.
The architecture will be explained, using the scenarios in section \ref{chap:problem:scenarios}.

For the user to be able to perform operations, he first must authenticate himself to the device. The device will come from fabric with a default authentication \ac{PIN}. To authenticate himself to the device, the user sends a \ac{PIN} through the software, the device then compares it to the authentication number stored inside. To protect the number it can be either stored in the secure eNVM if there is enough space, or in other non-volatile memory, signed with the device's public key. Once authenticated, the session will be unlocked, and the user will be able to perform operations.
If only the entity is authenticated, not the individual, a single \ac{PIN} is used for all authentications.
If instead the individual is authenticated, the user will send the registered name and the number, and the device will use the name to identify the correct number.

After authentication, the operations that can be performed, are split in three types:
\begin{itemize}
    \item Administration operations configure authentication and communication parameters;
    \item The secure communication operations provide the cryptographic services to secure communications;
    \item Communication management operations manage the keys used to secure connections.
\end{itemize}

% -----------------------------------------------------
\subsection{Administration}\label{chap:arch:ops:admin}

The administration operations will allow the user to manage the authentication related parameters.
For the first scenario, the only operations of this type is to change the authentication \ac{PIN}. The user sends the number to the device, and it will be securely saved inside it. The device will be initialized from fabric with a default \ac{PIN} which must be supplied to the user. Before performing any operation the user should change his PIN to begin secure communications.

The second scenario has an administrator role and a user role. Each user has its own \ac{PIN}. The administrator, beyond changing the authentication number, can register new users. To register a new user, both administrator and new user need to be physically together with the device. The administrator authenticates himself to the device, begins the registration process and allows the user to insert their name and \ac{PIN}. After this the user can login with name and number, and access all secure communication operations, as well as changing their own \ac{PIN}.

% -----------------------------------------------------
\subsection{Secure Communication}\label{chap:arch:ops:comms}

The main operations will be responsible to secure the communications between users. These operations will grant the confidentiality, integrity, authentication and non-repudiation services to communications.

Secure communications with \textbf{confidentiality} and \textbf{authentication}. The objective of this operation is to send and receive data to and from the device. The user sends plaintext data, and the device returns it encrypted and authenticated with the symmetric key stored inside the device, of the corresponding connection. In the equivalent operation reversed, the ciphertext is sent to the device, and the plaintext is returned;

\begin{figure}[h]
	\centering     %%% not \center
	\subfigure[Encrypt and authenticate data with K1 key]{\label{fig:arch-encrypt}\includegraphics[width=1\textwidth]{./Images/arch-encrypt.png}}
	\subfigure[Decrypt data and verify authentication with K1 key]{\label{fig:arch-decrypt}\includegraphics[width=1\textwidth]{./Images/arch-decrypt.png}}
	\caption{Procedure to secure data with authentication and confidentiality}
\end{figure}

Beginning in figure~\ref{fig:arch-encrypt}, the plaintext data sent to the device is first encrypted with the symmetric key \textit{K1}. Then a \ac{MAC} is generated from the encrypted data, using the symmetric key. Both fragments of information are returned to the user, and the user can then send it to other entities in possession of the used \textit{K1} key.
Following with figure~\ref{fig:arch-decrypt}, when a device receives the ciphertext, it decrypts the data using the same \textit{K1} key. Then it computes the MAC of the received \ac{IV} appended with the decrypted data. If the computed and received \ac{MAC} are identical, the data is authenticated.

\begin{figure}[h]
	\centering     %%% not \center
	\subfigure[Alice generates signature of a document]{\label{fig:arch-ds}\includegraphics[width=0.7\textwidth]{./Images/arch-ds.png}}
	\subfigure[Bob verifies the signature of the document]{\label{fig:arch-ds-verify}\includegraphics[width=0.7\textwidth]{./Images/arch-ds-verify.png}}
	\caption{Procedure to generate and verify a qualified digital signature}
\end{figure}

Qualified digital signatures provide \textbf{non-repudiation} to a piece of data, using the private key generated inside the box. The user sends the data to the box, and the subsequent signature will be returned. As pictured in figure~\ref{fig:arch-ds}, it is generated by calculating the hash of the data, and signing the digest with the device's private key. To verify a signature, illustrated in figure~\ref{fig:arch-ds-verify}, the device decrypts the hash with the signer's public key. Next, it computes the hash of the original document, and compares both hashes. If they are identical, the qualified signature is valid.

% -----------------------------------------------------
\subsection{Communication Management}\label{chap:arch:ops:key}

These operations will manage the needed keys to support secure communications, digital signatures and create new secure connections.
Supported key management operations are: symmetric key generation, symmetric key revocation, if communications are suspected to be compromised, and importation of other entities' keys.
These operations are only available in the scenario where each device has a pair of asymmetric keys.

\begin{figure}[h]
	\centering     %%% not \center
	\subfigure[Alice generates key]{\label{fig:arch-new-key}\includegraphics[width=0.8\textwidth]{./Images/arch-new-key.png}}
	\subfigure[Bob saves key]{\label{fig:arch-save-key}\includegraphics[width=0.8\textwidth]{./Images/arch-save-key.png}}
	\caption{Alice generates a new key to share with Bob}
\end{figure}

An entity receives their device, prepared to communicate with other entities, and a list of information of other entities, available to create communications. This information, namely the entities public key, needs to be imported into the device's non-volatile memory. It does not need to be secure storage.
After the key is imported, a secure connection with a new entity can be established, by sharing symmetric keys.
For an entity to communicate with a new entity, their device will generate a new symmetric key and store it either in secure storage, or in non-volatile memory, encrypted with the device's public key. Figure~\ref{fig:arch-new-key} illustrates the procedure Alice's device goes through to share a new key with Bob. The key is first signed with the sender's private key, which is stored in the device's secure storage, and then ciphered with the receiver's public key, which was imported before. Portrayed in figure~\ref{fig:arch-save-key}, Bob's device decrypts the message using its private key, and decrypts it again with Alice's public key. Finally the key is securely stored in the device, and both entities can subsequently establish secure communications.

When a symmetric key is revocated, due to reaching its expiration date, or from being compromised, entities can generate a new one, using the aforementioned procedure.
