% #############################################################################
% Abstract Text
% !TEX root = ../main.tex
% #############################################################################
% use \noindent in first paragraph
\noindent Nowadays, computers are exposed to a wide range of attacks. Individuals with high responsibility jobs, such as government officials and top company executives, are high profile targets to attacks. Successful attacks can compromise communications and leak sensitive information. Cryptographic keys are essential to the security of communications. Keys are usually stored in the user's computer, along with other sensitive information. The goal of this work was to isolate the user's computer from the system responsible for storage and management of keys. This was achieved by introducing a low cost, secure and portable \ac{HSM}. This work studied and evaluated the services of a SmartFusion2 board from Microsemi, serving as a HSM. Its security features and limitations were studied, and their performance was modelled. The board provides a robust set of security services: AES encryption, a hashing function, HMAC, some ECC primitives, a true random number generator, tamper detection and zeroization. A proof of concept system was developed on the device, focused on providing authentication and confidentiality to communications, using its services. The device acts as an intermediary between the owner, and other communicating users. The developed service, continuously encrypts and authenticates data, using the device's AES accelerator and a HMAC software implementation, with 256 bit security. A key management service was developed, which takes advantage of the secure storage, mitigates its limitations and allows regular key updates. The developed system also provides shared key generation with ECDH. The board is well-equipped to function as an HSM, but has some memory, performance and service limitations. This work provides the necessary groundwork and analysis for future work, using the SmartFusion2 as an HSM.
