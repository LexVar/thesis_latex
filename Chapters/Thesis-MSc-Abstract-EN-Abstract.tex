% #############################################################################
% Abstract Text
% !TEX root = ../main.tex
% #############################################################################
% use \noindent in first paragraph
\noindent Nowadays, computers are exposed to a wide range of attacks. Individuals with high responsibility jobs, such as government officials and top company executives are high profile targets. Successful attacks can compromise communications and leak sensitive information. Passwords and cryptographic keys are essential to the security of communications. The goal of this work was to isolate the user's computer from the system responsible for storage and management of keys. This was achieved by introducing a low cost, secure and portable \ac{HSM}. This work studied and evaluated the services of the SmartFusion2 \ac{SoC}, as a HSM. The SmartFusion2 provides a robust set of security services: symmetric key encryption, a hashing function, authentication, public-key cryptography primitives, a true random number generator, tamper detection and zeroization. The performance and limitations of these services was evaluated and modelled. A proof of concept was also developed on the device, focused on providing authentication and confidentiality to communications. The device continuously encrypts and authenticates communications as an intermediary between the owner, and other communicating users. A key management service is also provided, which takes advantage of the secure storage, given the device limitations, and allowing for regular key updates. The developed system also provides shared key generation. The board is well-equipped to function as a HSM, but has some feature, memory and performance limitations herein analysed. This work provides the necessary groundwork and analysis for future work, using the SmartFusion2 as a HSM.
