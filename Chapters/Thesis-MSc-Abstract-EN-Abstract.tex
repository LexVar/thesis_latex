% #############################################################################
% Abstract Text
% !TEX root = ../main.tex
% #############################################################################
% use \noindent in first paragraph
\noindent Nowadays, computers are exposed to a wide range of attacks. Individuals with high responsibility jobs, such as government officials and top company executives, are high profile targets to attacks. Successful attacks can compromise communications and leak sensitive information. Passwords and cryptographic keys are essential to the security of communications. The goal of this work is to isolate the user's computer from the system responsible for storage and management of keys. This was achieved by introducing a low cost, secure and portable \ac{HSM}. This work studied and evaluated the services of the SmartFusion2 \ac{SoC}, as the HSM of this system. The board provides a robust set of security services: AES encryption, a hashing function, HMAC, some ECC primitives, a true random number generator, tamper detection and zeroization. These services' performance was modelled and their limitations are presented. A proof of concept system was developed on the device, focused on providing authentication and confidentiality to communications. The device continously encrypts and authenticates communications as an intermediary between the owner, and other communicating users. A key management service was developed, which takes advantage of the secure storage, mitigates its limitations and allows regular key updates. The developed system also provides shared key generation with ECDH. The board is well-equipped to function as an HSM, but has some feature, memory and performance limitations. This work provides the necessary groundwork and analysis for future work, using the SmartFusion2 as an HSM.
