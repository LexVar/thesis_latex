% #############################################################################
% RESUMO em Português
% !TEX root = ../main.tex
% #############################################################################
% use \noindent in first paragraph
\noindent Hoje em dia, os computadores estão expostos a uma variedade de ataques. Indivíduos com cargos de grande responsabilidade, como membros do governo ou executivos de empresas, são alvos valiosos. Ataques bem sucedidos podem comprometer a segurança das comunicações e levar a fugas de dados. Chaves cryptográficas são essentiais para segurança das comunicações, e normalmente são guardadas no computador do utilizador. O objetivo deste trabalho é separar o computador do sistema responsável pelo armazenamento de chaves. Este objetivo foi alcançado introduzindo um módulo de hardware (HSM), seguro, portátil e de baixo custo. Foram analisados os serviços da placa SmartFusion2 da Microsemi, utilizada como um HSM, assim como as suas limitações e o seu desempenho. A placa providencia uma gama robusta de serviços de segurança: cifra AES, uma função de hash, HMAC, primitivas de ECC, um gerador de números aleatórios, deteção de interferências físicas e \textit{zeroization}. Um protótipo, que proporciona confidencialidade e autenticação às comunicações, foi desenvolvido no dispositivo. Este funciona como um intermediário entre o dono, e os outros utilizadores, que cifra e autentica dados continuamente. Um serviço de chaves foi desenvolvido, que utiliza a memória protegida da placa, e permite a atualização de chaves regularmente. O sistema também permite a geração de chaves partilhadas com ECDH. A placa está bem equipada para ser utilizada como HSM, no entanto tem algumas falhas e limitações nos seus serviços, memória e desempenho. Este trabalho fornece um sistema base e a análise essencial, para o desenvolvimento de projetos, utilizando a SmartFusion2 como HSM.
