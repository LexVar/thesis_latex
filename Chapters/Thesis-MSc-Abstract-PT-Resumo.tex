% #############################################################################
% RESUMO em Português
% !TEX root = ../main.tex
% #############################################################################
% use \noindent in first paragraph
\noindent Hoje em dia, computadores estão expostos a uma variedade de ataques. Indivíduos com cargos de grande responsabilidade, como membros do governo ou executivos de empresas, são alvos valiosos. Ataques bem sucedidos podem comprometer a segurança das comunicações e levar a fugas de dados. Passwords e chaves criptográficas são essenciais para a segurança das comunicações, e são normalmente guardadas no computador do utilizador. O objetivo deste trabalho é separar o computador do sistema responsável pelo armazenamento de chaves. Este objetivo foi alcançado introduzindo um módulo de hardware (HSM), seguro, portátil e de baixo custo. Este trabalho estudou e analisou os serviços da placa SmartFusion2 \ac{SoC}, como o HSM deste sistema. A placa providencia uma gama robusta de serviços de segurança: cifra AES, uma função de hash, HMAC, primitivas de ECC, um gerador de números aleatórios, proteções físicas contra ataques e \textit{zeroization}. A performance destes serviços foi modelada, e as suas limitações apresentadas. Além disso, um protótipo foi desenvolvido no dispositivo, que proporciona confidencialidade e autenticação a comunicações. O dispositivo cifra e autentica as comunicações continuamente, e haje como um intermediário entre o dono, e outros utilizadores. Um serviço de gestão de chaves também é providenciado, que tira partido da memória protegida da placa, e permite a atualização de chaves regularmente. O sistema também permite a geração de chaves partilhadas com ECDH. A placa está bem equipada para ser utilizada como HSM, no entanto foram identificadas algumas limitações nos seus serviços, memória e desempenho. Este trabalho fornece a análise essencial e um sistema base robusto para a utilização da SmartFusion2 como HSM.
