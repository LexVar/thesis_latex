% #############################################################################
% This is Chapter 1
% !TEX root = ../main.tex
% #############################################################################
% Change the Name of the Chapter i the following line
\fancychapter{Introduction}
\cleardoublepage
% The following line allows to ref this chapter
\label{chap:intro}

In the modern world, most people have access to a computer, involved in many everyday tasks, such as, web browsing, communications, social networks, news, entertainment, among many others.
There is no limit to what you can achieve with the Internet, using just a computer.
For this reason, computers have a wide range of attacks potentially exploitable by hackers, by taking advantage of software vulnerabilities or user mistakes.
This is of great concern to people with high responsibilities from their jobs, who deal with sensitive information, such as, government officials, top level company executives and diplomats.
Suffering an attack to a personal computer can be highly damaging as it can carry severe consequences for companies and countries.
In addition, high profile officials who deal with sensitive information are more likely to be targeted by attackers.
% Introduce concrete example of hack

% -----------------------------------------------------
% -----------------------------------------------------
\section{Problem}\label{chap:intro:problem}

New attacks, targeting computers, are discovered daily.
They can come from zero-day vulnerabilities, phishing scams and many others, the opportunities are endless. It is impossible to predict and protect against all.
Communications security, depends on the cryptography keys and passwords used. These are usually stored, along with other sensitive information, in the user's computer.
Instead of storing the data in the computer, a more optimal solution, meaning, harder to compromise the security of communications, is to separate the platform used by the user for communications (their computer), and the device responsible for managing, securing communications and storing sensitive data.
The goal is to add another layer of security, to make it difficult to compromise security even if the user's computer is compromised.
A secure and independent solution is needed to establish secure channels of communication, store keys and perform critical operations, even if the computer might be compromised.
A possible approach is the utilization of a personal physical device that is responsible for storing digital keys and perform critical operations.
These devices need to be highly secure and independent from the user's personal computer.

% -----------------------------------------------------
% -----------------------------------------------------
\section{Requirements}\label{chap:intro:requirements}

In order to address the problem and using the discussed approach, the implemented solution will have several requirements, to allow secure communications between multiple entities.
It should perform all critical operations to the security of the communications, as well as, store all relevant secrets to the security of the interactions.
A design requirement of the system is it should be easily usable to the regular user, with no technological expertise. The system must be efficient and low-cost, as so it is more easily accessible and scalable by interested users.

% TODO - FINISH PARAGRAPH EXPLAINING DOCUMENT STRUCTURE
% -----------------------------------------------------
% -----------------------------------------------------
\section{Document Structure}\label{chap:intro:doc}

This first chapter introduces the context, the problem and basic requirements of the system.
The second chapter will cover the technical background needed to comprehend the solution and state of the art.
The third chapter goes into detail about the problem, its entities, devices, extensive requirements, and possible use case scenarios.
% The fourth chapter will give context on the related work and existing solutions to the presented problem.
The fourth chapter covers the architecture of the solution including components and the operations.
The fifth chapter covers the developed protocols and implementation. The sixth chapter evaluates the system's performance and fulfilled requirements. The last chapter outlines the conclusions of the developed system and provides guidelines for future work.
