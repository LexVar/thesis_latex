% #############################################################################
% This is Chapter 2
% !TEX root = ../main.tex
% #############################################################################
% Change the Name of the Chapter i the following line
\fancychapter{Problem Definition}
\cleardoublepage
% The following line allows to ref this chapter
\label{chap:problem}

% TODO - ADD PARAGRAPH EXPLAINING SECTION STRUCTURE
This section will start by explaning the context surrounding the problem.
Next, the profile of the target clients will be described, and some examples will be given.
The next section contains a compilled list of relevant requirements for the solution, with the potential clients in mind.
Then it will shed the ligh on the essential concepts to understand the operations that need to be implemented. It will end by defining those operations, in accordance with the previously defined client requirements.

% -----------------------------------------------------
% -----------------------------------------------------
\section{Context} \label{chap:problem:context}

As discussed before, the same computers commonly used for communications and information storage are exploitable by attackers, and can cause a minor inconveniences, to possibly severe repercussions, such as, losing your confidential data to malicious parties.

An interesting approach to improve security is to add another layer of security to confidential data and communications through the addition of an device, independent of the user's personal computer. The device is responsible for the security of sensitive data and communications.

\subsection{Entities} \label{chap:problem:entities}

These type of devices are especially relevant to people high responsibility jobs, that handle very sensitive information, which have dire consequences if they are lost, corrupted or leaked.
Some examples are government officials who handle confidential information pertaining to a country, company executives, such as the CEO who have access to company secrets, diplomats who manage confidential treaties, and military officers who have access to information critical to a countries' security.

% Talk about groups and individuals
Additionally, not just individuals have interest in these systems, a device can be assigned to a group of people representing an entity. For example, in the armed forces, a device can be assigned to the navy, one to the infantry, and every other faction. Any ranked officer, or people with a certain level of authority, could use the entity device, to communicate with other people or entities, in behalf of the group.

\subsection{Devices} \label{chap:problem:devices}
% Devices
There are currently on the market some dedicated devices designed to secure communications and save private data.
These type of devices have physical tamper-resistant measures against tampering by attackers who wish to read the user's data. They also provide fail-safe mechanisms in case of an attack.
Hardware Security Modules (HSM) are high grade devices, with more computational power and larger storage capacity for the user's secrets.

Smart Cards, provide secure and portable tamper-resistant storage.
They have lower processing power, and smaller memory which only allows to store a small amount of data.
They have a low-cost, so can be produced in bulk and easily replaced. Only an RFID card reader is needed to read its information, and verify the owner's identity.
Because of these features, they are widely used in the retail, healthcare, communication and government industries.


% -----------------------------------------------------
% -----------------------------------------------------
\section{Requirements} \label{chap:problem:requirements}

In order to address the problem and using the discussed approach, the implemented solution must comply with the following requirements:
\begin{itemize}
	\item The box must be tamper-resistant, secure and personal;
	\item The system must allow secure interaction between different entities;
	\item All the user's secrets, such as keys and passwords must be stored in the device;
	\item The user's secrets must never be exposed to the outside;
	\item All critical operations must be performed in the device;
	\item The system must authenticate the user before performing operations. The system does not need to authenticate itself to the user;
	\item The system must be easy to use to the regular non-savvy user;
	\item The system should perform the operations in a reasonable time to minimize the user's wait;
	\item It should be relatively low cost.
\end{itemize}

% -----------------------------------------------------
% -----------------------------------------------------
\section{Concepts} \label{chap:problem:concepts}

In this section some necessary concepts will be explained in order for non-technical users can easily understand the background.

Crucial services to safeguard communications security are:
Confidentiality is a security service which keeps the contents of communications secret, except from the authorized parties.

Integrity safeguards communications from modifications by attackers.

The authentication service can verify the identity of any party, taking part in the communications.

Finally the non-repudiation service prevents an entity from denying authorship of a piece of information.

Cryptographic keys are an essential part of granting the aforementioned services. Users in possession of the keys can secure and access their messages.

Symmetric keys, in possession of all communicating parties, are used to secure messages and documents.
Asymmetric key pairs (public and private key), are used to enable communicating by for example, sharing new symmetric keys between users who wish to communicate. Secondly, they provide non-repudiation through digital signatures.

Digital signatures are a digital version of handwritten signatures, commonly used anywhere forgery detection is essential, for instance in financial transactions.
Qualified signatures are a special type of signatures where the private keys are generated and stored inside a device, such as a Smart Card, and never leave it.
This strong signature legally represents a person or a group. This type of signatures are used in the Portuguese Citizen Card.

% -----------------------------------------------------
% -----------------------------------------------------
\section{Services} \label{chap:problem:services}

% TODO - more high level

The solution will provide several services, which fulfill the requirements defined in \ref{chap:intro:requirements}.

\begin{itemize}
    \item Secure Storage to save the user's sensitive cryptographic keys, passwords, documents or any data;
    \item Key management, generation, revocation and importation of keys when the users deem necessary;
    \item Confidentiality to keep the contents of the communications secret, except from authorized entities;
    \item Integrity to safeguard communications from unauthorized modifications;
    \item Authentication to ascertain the identity of the data sender;
    \item Non-repudiation to prevent an entity from denying authorship of a piece of information.
\end{itemize}
