% #############################################################################
% This is Chapter 2
% !TEX root = ../main.tex
% #############################################################################
% Change the Name of the Chapter i the following line
\fancychapter{Problem Definition}
\cleardoublepage
% The following line allows to ref this chapter
\label{chap:problem}

% TODO - ADD PARAGRAPH EXPLAINING SECTION STRUCTURE

% -----------------------------------------------------
% -----------------------------------------------------
\section{Problem} \label{chap:problem:problem}




% -----------------------------------------------------
% -----------------------------------------------------
\section{Entities} \label{chap:problem:entities}

These type of devices are aimed at people that handle very sensitive information, which have dire consequences if they are lost, corrupted or leaked.
Some examples are government officials who handle confidential information pertaining to a country, company executives, such as the CEO who have access to company secrets, diplomats who manage confidential treaties, and military officers who have access to information critical to a countries' security.

% TODO - talk about devices

% -----------------------------------------------------
% -----------------------------------------------------
\section{Requirements} \label{chap:problem:requirements}

In order to address the problem and using the discussed approach, the implemented solution must comply with the following requirements:
\begin{itemize}
	\item The box must be tamper-resistant, secure and personal;
	\item The system must allow secure interaction between different entities;
	\item All the user's secrets, such as keys and passwords must be stored in the device;
	\item The user's secrets must never be exposed to the outside;
	\item All critical operations must be performed in the device;
	\item The system must authenticate the user before performing operations. The system does not need to authenticate itself to the user;
	\item The system must be easy to use to the regular non-savvy user;
	\item The system should perform the operations in a reasonable time to minimize the user's wait;
	\item It should be relatively low cost.
\end{itemize}

% -----------------------------------------------------
% -----------------------------------------------------
\subsection{Concepts} \label{chap:problem:concepts}
% TODO
% How to protect data
% how to ensure security of communications
	% Confidentiality
	% Integrity
	% Non-repudiation
	% authenticity
% Keys

% -----------------------------------------------------
% -----------------------------------------------------
\section{Services} \label{chap:problem:services}

% TODO - more high level

The solution will provide several services, which fulfill the requirements defined in \ref{chap:intro:requirements}.

\begin{itemize}
    \item Secure Storage to save the user's sensitive cryptographic keys, passwords, documents or any data;
    \item Key management, generation, revocation and importation of keys when the users deem necessary;
    \item Confidentiality to keep the contents of the communications secret, except from authorized entities;
    \item Integrity to safeguard communications from unauthorized modifications;
    \item Authentication to ascertain the identity of the data sender;
    \item Non-repudiation to prevent an entity from denying authorship of a piece of information.
\end{itemize}
