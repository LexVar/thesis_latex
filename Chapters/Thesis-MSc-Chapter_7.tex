% #############################################################################
% This is Chapter 7
% !TEX root = ../main.tex
% #############################################################################
% Change the Name of the Chapter i the following line
\fancychapter{Conclusion}
\cleardoublepage
% The following line allows to ref this chapter
\label{chap:conclusion}

% -----------------------------------------------------
% -----------------------------------------------------
% \section{Summary} \label{chap:conclusion:summary}
This work focused on evaluating the security features of the Microsemi smartfusion2 board, modeling its performance, and developing a prototype to secure communications between devices.
The system provides a varied array of security services using symmetric keys, AES with 128 and 256 bit key encryption, SHA-256, HMAC authentication with 256 bit SHA and a SHA based key derivation function. For asymmetric cryptography it offers ECDH for key generation, and additional ECC primitives, with the P-384 NIST defined curve, which can be used to implement digital signatures. Lastly, a true random number generator, a PUF based secure storage solution, tamper detection capabilities with several flags, and a zeroization feature with multiple recoverabity options.
The prototype focused on the implementation of a service which provides authentication and encryption, using symmetric keys. The system is able to encrypt up to 36 KB of data using the AES and HMAC cores, with adequate 256 bit security. This limitation is imposed, due to the limited 80 KB of RAM, without error correction and detection, and the absence of support for continuous AES encryption and MAC generation. This hurdle was overcome by using an HMAC software implementation, and taking advantage of AES CBC mode, which uses the last ciphertext block as the IV for the next block. With this changes, a continuous authenticated encryption service was developed. The optimal buffer size, maximizing the available RAM space and service speed was studied, and its conclusions presented.
A key generation service using asymmetric key pairs was also implemented. It generates a shared secret, using an internal private key and a public key. The shared secret is used to derive a new symmetric key, using a salt value.
The system also has a limitation in its ECC primitives. There is no direct API for digital signatures, it can multiply a scalar with a point, enabling ECDH and generation of a public key from a private key. It can add two points on the curve and contains the curve's base point value. It cannot generate a random private key, or digital signatures, with ECDSA, without a big integer library. Without this, private keys must be generated outside the device and imported. The inclusion of such a library is complex due to the RAM memory limit, even disabling error correction and detection, and this librarires tend to be heavy in code space. 
Regarding the PUF service for storing keys, it has 56 slots for a maximum of 512 bytes each. This service uses private blocks of the eNVM to store part of the information for regeneration of the keys stored in the PUF. Therefore this service suffers from the same write cycle problem of the eNVM. It is limited to 1000 write cycles for a predicted twenty year lifespan. To mitigate this, a key management service was developed. It uses a key encryption key stored in a PUF slot, to encrypt and authenticate the set of keys stored in a non-volatile memory. The generated MAC of the stored keys is also stored in a PUF slot. A set of keys can be imported, with only one write to PUF done each time. The smartfusion2 board also allows access to external storage, which can be used to store the encrypted set of keys.

This work contributes with an extensive characterization of the smartfusion2 device. It studied each security service advantage and possible tradeoffs. It provides a blueprint of possible utilizations and avenues to follow, to use this services. Furthermore, it modeled the performance of every service, providing a usefull prediction of the behaviour to expect from the system. Lastly, the implemented prototype provides solid groundwork for a secure communications service, in a low cost HSM type device.

% -----------------------------------------------------
% -----------------------------------------------------
\section{Future Work} \label{chap:conclusion:future-work}

% Secure the communication channel the information is sent through.
